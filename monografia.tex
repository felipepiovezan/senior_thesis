\documentclass{ufscThesis}
\newcommand{\ABNTbibliographyname}{REFERÊNCIAS} % Necessário para abnTeX 0.8.2

\instituicao[a]{Universidade Federal de Santa Catarina}
\departamento[a]{Departamento de Informática e Estatística}
\curso[o]{Curso de Bacharelado em Ciência da Computação}
\titulo{Análise da eficiência energética de algoritmos de criptografia baseados em curvas elípticas}
\autor{Felipe de Azevedo Piovezan}
\grau{Bacharel em Ciência da Computação}
\local{Florianópolis - Santa Catarina}
\data{02}{Dezembro}{2015}
\orientador[Orientador]{Prof. Dr. Luiz Cláudio Villar dos Santos}
\coorientador[Coorientador]{Prof. Dr. Daniel Santana de Freitas}
\coordenador[Coordenador]{Prof. Dr. Renato Cislaghi}

\numerodemembrosnabanca{5} % Isso decide se haverá uma folha adicional
\orientadornabanca{sim} % Se faz parte da banca definir como sim
\coorientadornabanca{sim} % Se faz parte da banca definir como sim
%\bancaMembroA{Prof. Dr. Luiz Cláudio Villar dos Santos} %Nome do presidente da banca
\bancaMembroB{Prof. Dr. José Luís Almada Güntzel}      % Nome do membro da Banca

\agradecimento{
Um agradecimento
}

\epigrafe{Nevermore}
{The Raven}

\textoResumo {
Texto Resumo
}
\palavrasChave {Palavras chave}

\textAbstract {
text abstract
}
\keywords {keywords}

\begin{document}

%--------------------------------------------------------
% Elementos pré-textuais
%\capa  
\folhaderosto[comficha] % Se nao quiser imprimir a ficha, é só não usar o parâmetro
\folhaaprovacao
\paginadedicatoria
\paginaagradecimento
\paginaepigrafe
\paginaresumo
\paginaabstract
%\pretextuais % Substitui todos os elementos pre-textuais acima
\listadefiguras % as listas dependem da necessidade do usuário
\listadetabelas 
\listadeabreviaturas
\listadesimbolos
\sumario
%--------------------------------------------------------
% Elementos textuais

\chapter{Introdução}
Introducao!

\end{document}
